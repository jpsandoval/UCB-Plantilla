%-=-=-=-=-=-=-=-=-=-=-=-=-=-=-=-=-=-=-=-=-=-=-=-=-=-=-=-=-=-=-=-=-=-=-=-=-=-=-=-=-=-=-=-=-=-=-=-=-=-=-=-=-=-=
% UCB CBBA
% DEPARTAMENTO DE CIENCIAS EXACTAS Y INGENIERIA
% PLANTILLA PARA DOCUMENTOS ACADEMICOS
% Juan P. Sandoval
%-=-=-=-=-=-=-=-=-=-=-=-=-=-=-=-=-=-=-=-=-=-=-=-=-=-=-=-=-=-=-=-=-=-=-=-=-=-=-=-=-=-=-=-=-=-=-=-=-=-=-=-=-=-=

\documentclass[upright, contnum, hidelinks,times]{ucbcba}

\depto{Departamento de Ingenier\'ia y Ciencias Exactas}
\author{Nombre Estudiante}
\title{\textbf{Titulo de Tesis}}

\date{2018}
\tipo{Proyecto de Grado }
\carrera{Ingenier\'ia de Sistemas}

\usepackage[utf8]{inputenc}
% descomentar para espa�ol
%\usepackage[spanish]{babel}
\usepackage{gensymb}
\usepackage{fancyhdr}
\usepackage[toc,page]{appendix}
\usepackage{amsmath}
\usepackage{lipsum}
\usepackage[font=footnotesize, labelfont=bf]{caption}
\usepackage{anyfontsize}
\usepackage[pscoord]{eso-pic}
\usepackage{listings}
\usepackage{sectsty}
\usepackage[numbers,sort]{natbib}
\usepackage{titlesec}
\usepackage[nottoc,notlot,notlof]{tocbibind}
\usepackage{xcolor}
\usepackage[numbers,sort]{natbib}
\usepackage[nottoc,notlot,notlof]{tocbibind}
\usepackage{xspace}
\usepackage{ifthen}
\usepackage{amsbsy}
\usepackage{url}
\usepackage{amssymb}
\usepackage{booktabs}
\usepackage{graphicx}
\usepackage{verbatim}
\usepackage{balance}
\usepackage{multirow}
\usepackage{needspace}
\usepackage{microtype}
\usepackage{bold-extra}
\usepackage{subfigure}
\usepackage{wrapfig}
\usepackage{textcomp}
\usepackage{listings}



\newcommand{\bibtexdb}{referencias}
\newcommand{\placetext}[3]{% \placetextbox{<horizontal pos>}{<vertical pos>}{<stuff>}
  \setbox0=\hbox{#3}% Put <stuff> in a box
    \AddToShipoutPictureFG*{% Add <stuff> to current page foreground
        \put(\LenToUnit{#1\paperwidth},\LenToUnit{#2\paperheight}){\parbox{\textwidth}{#3}}%
    }%
}%
\newcommand{\placetextbox}[3]{% \placetextbox{<horizontal pos>}{<vertical pos>}{<stuff>}
  \setbox0=\hbox{#3}% Put <stuff> in a box
    \AddToShipoutPictureFG*{% Add <stuff> to current page foreground
        \put(\LenToUnit{#1\paperwidth},\LenToUnit{#2\paperheight}){\vtop{{\null}\makebox[0pt][c]{#3}}}%
    }%
}%


% Configuraci�n para los t�tulos de cada capitulo.
\renewcommand\chaptertitlename{CAP\'ITULO}
\titleformat{\chapter}[hang]
  {\normalfont\fontsize{12}{12}\selectfont\bfseries}
  {\MakeUppercase{\chaptertitlename}\ \thechapter:}
  {.0em}
  {\MakeUppercase}
\titlespacing*{\chapter}{0pt}{0pt}{0pt}

\chaptertitlefont{\fontsize{12}{12}\selectfont}
\sectionfont{\fontsize{12}{12}\selectfont}
\subsectionfont{\fontsize{12}{12}\selectfont}


% graphics: \fig{position}{percentage-width}{filename}{caption}
\DeclareGraphicsExtensions{.png,.jpg,.pdf,.eps,.gif}
\graphicspath{{figures2/}}
\newcommand{\fig}[4]{
	\begin{figure}[#1]
		\centering
		\includegraphics[width=#2\textwidth]{#3}
		\caption{\label{fig:#3}#4}
	\end{figure}}
\newcommand{\largefig}[4]{
	\begin{figure*}[#1]
		\centering
		\includegraphics[width=#2\textwidth]{#3}
		\caption{\label{fig:#3}#4}
	\end{figure*}}
\newcommand{\wrapfig}[5]{	
\begin{wrapfigure}{#1}{#2\textwidth}
  \begin{center}
    \includegraphics[width=#3\textwidth]{#4}
  \end{center}
  \caption{\label{fig:#4}#5}
\end{wrapfigure}}
% abbreviations
\newcommand{\ie}{\emph{i.e.,}\xspace}
\newcommand{\eg}{\emph{e.g.,}\xspace}
\newcommand{\etc}{\emph{etc.}\xspace}
\newcommand{\etal}{\emph{et al.}\xspace}
% lists
\newenvironment{bullets}[0]
	{\begin{itemize}}
	{\end{itemize}}
\newcommand{\seclabel}[1]{\label{sec:#1}}
\newcommand{\secref}[1]{Section~\ref{sec:#1}}
\newcommand{\figlabel}[1]{\label{fig:#1}}
\newcommand{\figref}[1]{Figura~\ref{fig:#1}}
\newcommand{\tablabel}[1]{\label{tab:#1}}
\newcommand{\tabref}[1]{Tabla~\ref{tab:#1}}
%Specialized macros
\newcommand{\myparagraph}[1]{\vspace{0.2cm}\noindent\textbf{\textit{#1}.}\xspace}


\clubpenalty = 100000
\widowpenalty = 100000
\displaywidowpenalty = 100000


\setcounter{chapter}{0}


\begin{document}

\frontmatter
\maketitle


% descomentar para agregar un resumen de tesis
%\begin{resumen}
%\lipsum[3-4]
%\end{resumen}


% descomentar para agregar un abstract en ingles
%\begin{abstract}
%\lipsum[3-4]
%\end{abstract}

% descomentar para agregar dedicatoria
%\begin{dedicatoria}
%Dedicado a mis padres...
%\end{dedicatoria}


% descomentar para agregar agradecimientos
%\begin{agradecimientos}
%\lipsum[1-2]
%\end{agradecimientos}

\renewcommand*\contentsname{\hfill Indice General \hfill}
\tableofcontents


\renewcommand\listtablename{Lista de Tablas}
\listoftables

\renewcommand\listfigurename{Lista de Figuras}
\listoffigures

\mainmatter

\begin{intro}


Las promesas de collections se crearon para optimizar la composicion de operaciones de colections. Las operaciones se pueden optimizar bajo un conjunto de reglas de inferencia \cite{juampi}.

Este es un ejemplo de citaci\'on \cite{clay_2009}.

\section*{Antecedentes}
\addcontentsline{toc}{section}{\protect\numberline{}Antecedentes}%



%para referenciar se pone el path de la imagen
Ejemplo de como hacer referencia a una imagen \figref{imagenes/UCB-Escudo}

% 0.5 significa que reducira el tama�o de la imagen a la mitad.
\largefig{h}{0.5}{imagenes/UCB-Escudo}{Ejemplo de Figura}




%se pone el ide de la tabla... el id es definido por el tag \label
Ejemplo de como hacer referencia a una tabla \ref{categorias}

%EJEMPLO DE TABLA

\begin{table}[t]
\centering
\begin{tabular}{llrrrr}
\hline
\multicolumn{2}{|l|}{\textbf{Cambios de C\'odigo Fuente}}                                                                                          
& \multicolumn{1}{r|}{\textbf{R}} 
& \multicolumn{1}{r|}{\textbf{I}} 
& \multicolumn{1}{r|}{\textbf{R/I}} 
& \multicolumn{1}{r|}{\textbf{Total}} \\ \hline
1	& Method call additions									& 	23 	& 	0  	& 	1	&	24 (29\%)  	\\
2	& Method call swaps										& 	15	& 	9 	&	0	&	24  (29\%)\\
 \hline
 
	& {\bf Total}											&	52	&	28	&	4	&	84 (100\%)	 \\
\end{tabular}
\caption{Ejemplo de Tabla}
\label{categorias}
\end{table}


\subsection*{Ejemplo de subsecci\'on}


\section*{Problema}
\addcontentsline{toc}{section}{\protect\numberline{}Antecedentes}%

\subsection*{\emph{Situaci\'on Problematica}}
\addcontentsline{toc}{subsection}{\protect\numberline{} \emph{Situacion Problematica}}%



\subsection*{\emph{Formulaci\'on del Problema}}
\addcontentsline{toc}{subsection}{\protect\numberline{} \emph{Formulaci\'on del Problema}}%



\section*{Objectivos}
\addcontentsline{toc}{section}{\protect\numberline{}Objectivos}%



\subsection*{Objetivo General}
\addcontentsline{toc}{subsection}{\protect\numberline{}Objetivo General}%


\subsection*{Objetivo Espec\'ificos}
\addcontentsline{toc}{subsection}{\protect\numberline{}Objetivo Espec\'ificos}%


\section*{Alcances}
\addcontentsline{toc}{section}{\protect\numberline{}Alcances}%



\section*{Limites}
\addcontentsline{toc}{section}{\protect\numberline{}Limites}%




\section*{Justificaci\'on}
\addcontentsline{toc}{section}{\protect\numberline{}Justificaci\'on}%


\section*{Cronograma}
\addcontentsline{toc}{section}{\protect\numberline{}Cronograma}%

%\section*{Subsecci\'on 1}
%\addcontentsline{toc}{subsection}{\protect\numberline{}Subsecci\'on 1}%


\end{intro}

%% !TEX root = thesis.tex
\chapter{TITULO DE CAPITULO UNO}

\section{Secci\'on A}
\subsection{Secci\'on BB}

%\begin{conclusions}
\label{chap_conclusions}
\end{conclusions}



\nocite{*}
\renewcommand\bibname{BIBLIOGRAF\'IA}
\bibliographystyle{plain}
\bibliography{\bibtexdb}

\renewcommand\appendixtocname{APENDICES}
\renewcommand\appendixpagename{APENDICES}
\begin{appendices}
mis apendices
\end{appendices}
\end{document}
